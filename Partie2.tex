\chapter{Script de comparaison de base de donnée}

N'ayant pas de notion de programmation orienté objet\, \footnote{ce qui sera
enseigner en deuxieme année} je n'ai pas pu rejoindre les developeur de
l'entreprise dans l'application qu'ils étaient occupés d'éffectuer, cela dit on
m'a confier une tache annexe qui est la comparaison des bases de données. Le
besoin premier de ce script est de consulter les différences de structure qui
existe entre une base de référence est une base à mettre à jour. Dans le cas de
l'entreprise, il permettrait un suivi des mises à jour des applications fournis
au client, et pour moi, m'initier à la programmation orienté objet\,
\footnote{La programmation orientée objet est un paradigme de programmation qui
consiste à utiliser des objets ; un objet représente un concept, une idée ou
toute entité du monde physique, comme une voiture, une personne ou encore une
page d'un livre.} des le premier stage.

\section{Le départ}

À l'aide d'un cours sur internet, j'ai commencé à créer ma première classe.
Cette classe après instanciation représenterait l'objet "base de donnée" sous
forme de tableau.  Passer de la programmation fonctionnelle\,
\footnote{La programmation fonctionnelle est un paradigme de
programmation qui repose sur l'utilisation majoritaire des fonctions et
procédures} à l'objet fut vraiment difficile. Me rendant compte que je bloquai
énormément, je fis des recherches sur internet pour trouver un programme
équivalent sur lequel je me suis appuyer pour commencer. M.Dubourg m'a mis sur
la voie en me disant d'utiliser des classes est des méthodes toute prête de
l'entreprise ce qui m'ôtas une grosse épine du pied car je n'avais aucune idée
de comment je devais faire pour transcrire une structure de base de donnée en
objet. Pour utiliser les sources de la société, j'ai du mettre en place un
répertoire de travail et récupérer les sources via leur ancien gestionnaire de
version de code source.  Ceci étant fait il s'agissait maintenant de réussir à
faire fonctionner le site web principal en local sur ma machine. Mon logiciel
MAMP\, \footnote{Macintosh Apache MySQL Php est une combinaison de logiciel}
m'affichant plein d'erreurs, j'ai décidé d'installer manuellement chacun des
logiciels présents dans celui-ci.  Même après cette manipulation le problème
n'avais pas disparu, mon tuteur de stage m'est venu en aide après de longues
heures à chercher en vain.  Le problème ce trouvai au niveau de la
configuration de PHP\, \footnote{\emph{Hypertext Preprocessor} est un langage
de scripts libre principalement utilisé pour produire des pages Web dynamiques}
qui affichai tout les avertissements de manière trop stricte, ce qui refusai
tout lancement de la page principale et aussi le fait que je devais vider le
cache de mon navigateur. De nombreuses heures de recherche juste à cause d'un
cache internet pas vidé fut extrêmement frustrant...

\section{Comparaison des tables}

%%% arrivé ici %%%

Il a fallut dans un premier temps que je recherche comment extraire le nom
d'une base de donnée ainsi que le nom de ses tables. La réponse a cette
question est dans la documentation de MySQL. Une base de donnée est fournit à
l'installation est qui s'apelle information schéma. En résumé, c'est une base
de donnée qui contient les autres bases de donnée.

Ensuite, j'ai transformer l'algorithme procédurale sur plusieurs semaines en objet, le fait de passer par cette étape intermédiaire m'as permis d'abord de résoudre le
problème algorithmique, puis après me consacrer sur la façon de programmer.

La requête récuperant les données utiles, il a fallut concevoir un algorithme capable de comparer les deux tableaux d'objets par rapport a leurs nom :
\begin{itemize}
    \item Si la base de donnée de référence à une table qui n'est pas dans la base de donnée a mettre à jour, on l'ajoute.
    \item Si la base de donnée de mise à jour contient une table qui n'est pas dans la base de donnée de référence, on l'enlève.
\end{itemize}

J'ai du revoir mon algorithme plusieurs fois car ma fonction de comparaison de chaine de caractère (strnatcmp) comparait les chaines de caractère comme un humain le ferait,
alors que mySQL trie les tables de manière binaire c'est a dire les codes Ascii des caractères ce qui fesait que la première fonction donner des resultats érronées dans des
cas particuliers.

Tri de chaînes standard
Array
(
    [0] => img1.png
    [1] => img10.png
    [2] => img12.png
    [3] => img2.png
)

Tri de chaînes "ordre naturel"
Array
(
    [0] => img1.png
    [1] => img2.png
    [2] => img10.png
    [3] => img12.png
)

\section{Comparaison des champs}

Maintenant que je me retrouvai dans des tables ayant les mêmes noms, il fallait que je compare le contenu, lui aussi en comparant le nom des champs, il m'as pas fallut
longtemps pour comprendre que l'algorithme était exactement le même, le plus dur étant de savoir comment j'allais faire pour me répeter le minimum possible ce qui a bloqué
énormement mon avancement pour pas grand chose. M.Dubourg constatant que je n'avançais plus ma conseiller de faire comme je le voyais dans ma tête plutôt que d'éssayer de
modifier l'algorithme dans ma tête en objet puis de l'écrire. Une fois le script fonctionnel mais très mal optimisé et peu lisible, celui-ci m'as guidé via des annotations
est des explications poussées sur la marche qui semblai la meilleur. De fil en aiguille le code source est passé d'environ trois-cent cinquante lignes à cent cinquante.

Ensuite légèrement different, je devais comparer le contenu des champs des tables lorsque les champs parcouru avait le même nom, je compare tout
simplement le contenu des champ pour connaitre l'issue finale, à savoir que je ne pouvais pas décider si les tables était similaire sans avoir balayer
tout les champs et toutes les caractèristique de ceux-ci.

Après tout ceci fini est optimisé, j'implémente une nouvelle fonctionnalité qui serai d'afficher un texte lisible qui énonce les differences des deux
bases donnée. En fonction des résultats obtenu, je génère des balises html et du texte pour que cela soit comprehensible à l'utilisateur. Ceci étant
fait assez rapidement je suis passé à la génération des requetes SQL permettant de mettre a jour la base de donnée comparé à la référence. C'est à ce
stade que j'ai constaté que je ne pourrais pas prendre en compte les clés étrangère dans mon code à moin de revoir totalement tout le script.
M.Dubourg ma rassurer sur le fait que l'entreprise n'utilisai pas le type de base de donnée MySLAM et que donc aucune de leurs table comportait de clé
étrangère, cependant les clés primaire concaténées reste problèmatique car pas imaginer pendant la conception. Nous avons décider que la création de
ce genre de cas ce fera à la main pour pas repartir de zéro.

Après avoir analyser mes methodes de génération de phrase lisible et de requetes SQL, M.Dubourg ma présenter un outils nommée "Smarty" qui permet de
dissocier la partie traitement de la partie affichage car il est vrai que mes méthodes était vraiment très sale. Pour résumer, j'écrivais des chaines
de caractère dans une seule variable en écrivant les noms des tables ou des champs, en concatenant le tout avec des balises html de saut de ligne un
peu nimporte où. J'ai du parcourir la doc de Smarty pour comprendre son fonctionnement, il s'avère que la synthaxe est très particulière mais très
éfficace. J'ai terminer par faire de l'agencement sur ma page pour que chaque ligne lisible par un néophyte soit en face de la ligne en SQL dans un
tableau a deux colonnes tout ceci avec l'aide de smarty.

A peu près au milieu du stage, nous nous sommes heurté a un problème technique. Le serveur de l'entreprise ne contenai pas de version réçente de Git.
En sachant que celui-ci est mutualisé, nous n'avions pas les autorisations nécessaire pour installer une nouvelle version. Les logiciels libres sont
réputé pour être rétro-compatibles mais en toutes logique il ne dispose pas des nouveautés sans les mettre à jour. Le problème ce situe sur le fait de
pouvoir modifier des branches distantes dites "de suivi". Cette fonctionnalité permet à plusieurs développeur de travailler en même temps sur une branche différente de
celle par defaut, par exemple pour experimenter de nouvelles choses à plusieurs. A ce moment là j'ai été très déçu et pensais mon travail
inutilisable. Cependant, mes éfforts n'ont pas été vain puisque M.Dubourg après des recherches à trouver sur internet un utilisateur ayant reussi à
installer git sur un serveur mutualisé. Nous n'avions peut-être pas les droits d'accès au dossiers des programmes mais rien ne nous empechai de
compiler git a partir des sources pour nous crée en locale le logiciel. J'étais déçu d'avoir passer la partie compilation dans mon livre UNIX.
Ceci étant fait, en redéfinissant la variable système qui stocke les chemains des applications utiliser notre compilé dernier cri.

accès à la partie installation des programmes
Mon travail suivant consistait a lister les dépôt Git qui sont en fin de compte les dossiers contenant les sources de l'entreprise mais versionnée
dans une page php. En effet le fait que le serveur OVH centralise les codes sources de la société, les developpeurs devait récuperer une copie de ses
dossiers pour pouvoir travailler dessus, le problème étant qu'il devait lancer FilleZilla pour connaitre les noms des dossiers Git puis recopier la
bonne url avec le bon nom est très fastidieux. J'ai donc à l'aide de mes connaissance en système UNIX crée un script en langage PHP qui analyse un
dossier défini et qui liste les differents dépôt tout en leur métant les bonnes addresses de téléchargement en préfixe. Du coup, un simple copier
collé du dépôt voulu puis l'importation avec tortoiseGit ce fait en un clic. Gain de productivité et de simplicité en somme.

Trois jours avant mon départ, le grand déménagement simpose. Toute la partie étude et confection du tutoriel git prend forme car j'ai basculer tout
les progets qui était au préalable sur Subversion vers Git tout en gardant l'historique des changement provenant de SVN. Après quelques recherche, une
commande poussé de Git ma permis de répondre a ce besoin.

Nous avons longuement discuté sur le côté technique de git. En effet cette outil est à la base un logiciel libre provenant de linux et n'as pas
d'interface graphique, ce choix est établi sur le fait qu'un développeur n'as pas forcement besoin de ça pour travailler (Sans nul doute que pour un
graphiste, une souris est un élément indispensable pour ses créations mais pas pour produire du code) et aussi que cette outil regorge de
fonctionnalité et donc très difficile à simplifier à travers une interface ne pouvant fonctionner qu'avec une souris. L'équipe n'étant pas très féru de
ligne de commande et le logiciel TortoiseGit reprenant que les fonctions éssentielle de Git, il a fallu que je conçois des petits programmes qui
fonctionne en un clic pour simplifier les choses répétitives.


%%-- Note de bas de page sur les stades
%\protect\footnote{Par exemple, on peut faire un pied de page : \begin{itemize} \item avec une liste à puces ; \item avec une liste à puces ; \item avec une liste à puces.
%\end{itemize} } -- Fin Note de bas de page sur les stades

\clearpage
