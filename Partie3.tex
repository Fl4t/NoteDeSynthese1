\chapter{Les utilitaires}

\section{Introduction} % (fold)

Nous avons longuement discuté sur le côté technique de git. En effet cette
outil est à la base un logiciel libre provenant de linux et n'as pas
d'interface graphique, ce choix est établi sur le fait qu'un développeur n'as
pas forcement besoin de ça pour travailler (Sans nul doute que pour un
graphiste, une souris est un élément indispensable pour ses créations mais pas
pour produire du code) et aussi que cette outil regorge de fonctionnalité et
donc très difficile à simplifier à travers une interface ne pouvant fonctionner
qu'avec une souris. L'équipe n'étant pas très féru de ligne de commande et le
logiciel TortoiseGit reprenant que les fonctions éssentielle de Git, il a fallu
que je conçois des petits programmes qui fonctionne en un clic pour simplifier
les choses répétitives.

% section Introduction (end)

\section{Lister les dépôts du serveur}

Mon travail suivant consistait a lister les dépôt Git dans une page php. En
effet, le fait que le serveur OVH centralise les codes sources de la société,
les developpeurs devait récuperer une copie de ses dossiers pour pouvoir
travailler dessus, le problème étant qu'il devait lancer FilleZilla pour
connaitre les noms des dossiers Git puis recopier la bonne url avec le bon nom
est très fastidieux. J'ai donc à l'aide de mes connaissance en système UNIX
crée un script en langage PHP qui analyse un dossier défini et qui liste les
differents dépôt tout en leur métant les bonnes addresses de téléchargement en
préfixe. Du coup, un simple copier-collé du lien du dépôt voulu dans
tortoiseGit suffit à cloner. Gain de productivité et de simplicité en somme.

\section{Mise à jour locale automatique} % (fold)

Les developpeurs possédent des clones des dépôts distants\, \footnote{Ils ce
trouvent tous sur le serveur OVH qui sert de point de rencontre comme nous
l'avons vu.} en locale pour bien évidament pour maintenir le code source. Le
fait qu'il y ai une multitude de dépôts implique qu'il faut mettre à jour les
clones locaux à chaque fois pour récuperer les modifications des autres
développeurs ce qui est très répétitif et fastidieux. J'ai donc crée un script
Batch\, \footnote{Désigne un fichier contenant une suite de commandes qui
seront traitées automatiquement par Windows \copyright.} qui parcours tout les
dossiers contenant les sources des dépôts et les mets a jour. Pour cela j'ai
parcouru la documentation de la console windows et ça n'as pas été évident du
tout. Pour l'anecdote, je devais à partir de mon MacBook\, \footnote{Système de
type UNIX.}, lancer une machine virtuelle windows\, \footnote{À partir du
logiciel VirtualBox.} pour crée mon script batch qui lance une console Linux
pour faire la mise à jour...

% section Mise à jour distante automatique (end)

\section{Mise en production automatique} % (fold)

Une fois que les développeurs sont satisfait de leurs modifications, il doivent
mettre à jour le dépots distant pour partager leurs travaux. Une fois ses
modifications validées par le chef de projet, celui-ci doit mettre en ligne sur
le dépôt de production. J'ai crée un script Batch pour que ça ce fasse en un
clic.

% section Mise en production automatique (end)
\clearpage
