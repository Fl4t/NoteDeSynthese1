\chapter{Gestion de version de code source}

\section{L'arrivé en entreprise}

Le premier jour en entreprise fut un peu spécial pour moi. En effet, mon
entretien s'étant passé un vendredi après-midi, seul les gérants de
l'entreprise était présent, du coup ça à été pour moi l'occasion de rencontrer
les deux developpeurs, Cyril (ancien BTS IG a jean bart) et Ludovic. M.Dubourg
ma imposer dans un premier temps de me documenter un maximum sur Git pour que
je maitrise globalement l'application et savoir si elle répondai bien aux
attentes.

Les problèmatiques principales sont les suivantes :
\begin{itemize}
    \item Peut-on fusionner les branches éfficacement ?
    \item Si oui, peut-on séléctionner les commits spécifiques à fusionner ?
\end{itemize}

Avant de pouvoir répondre à ses questions il me semble évident de comprendre le sens des mots qui n'était pas tous clair au départ.

\subsection{Git : le gestionnaire de version}

Les logiciels de gestion de versions sont utilisés principalement par les
développeurs. En effet, ils sont quasi exclusivement utilisés pour gérer des
codes sources, car ils sont capables de suivre l’évolution d’un fichier texte
ligne de code par ligne de code. Ces logiciels sont fortement conseillés pour
gérer un projet informatique.

Ils retiennent qui a effectué chaque modification de chaque fichier et
pourquoi. Ils sont par conséquent capables de dire qui a écrit chaque ligne de
code de chaque fichier et dans quel but ; si deux personnes travaillent
simultanément sur un même fichier, ils sont capables d’assembler (de fusionner)
leurs modifications et d’éviter que le travail d’une de ces personnes ne soit
écrasé.

Ces logiciels ont donc par conséquent deux utilités principales :
\begin{itemize}
    \item suivre l’évolution d’un code source, pour retenir les modifications effectuées sur chaque fichier et être ainsi capable de revenir en arrière en cas de problème ;
    \item travailler à plusieurs, sans risquer de se marcher sur les pieds. Si deux personnes modifient un même fichier en même temps, leurs modifications doivent pouvoir être fusionnées sans perte d’information.
\end{itemize}

\subsubsection{Les rudiments}

\begin{itemize}
    \item[le commit :] corréspond à un enregistrement des modifications dans le temps. Admettons un fichier contenant un paragraphe
    \item[la branche :] corréspond a une copie des fichiers sources par rapport a une branche principale
\end{itemize}


%-- Note de bas de page sur les stades
%\protect\footnote{Par exemple, on peut faire un pied de page :
%\begin{itemize}
%\item avec une liste à puces ;
%\item avec une liste à puces ;
%\item avec une liste à puces.
%\end{itemize}
}
%-- Fin Note de bas de page sur les stades

Ici du texte et du blabla, ce que l'on veut dire et écrire. A remplacer. Ici du texte et du blabla, ce que l'on veut dire et écrire. A remplacer. Ici du texte et du blabla, ce que l'on veut dire et écrire. A remplacer. Ici du texte et du blabla, ce que l'on veut dire et écrire. A remplacer. Ici du texte et du blabla, ce que l'on veut dire et écrire. A remplacer. Ici du texte et du blabla, ce que l'on veut dire et écrire. A remplacer.

\begin{itemize}
\item avec une liste à puces ;
\item avec une liste à puces ;
\item avec une liste à puces.
\end{itemize}

Ici du texte et du blabla, ce que l'on veut dire et écrire. A remplacer. Ici du texte et du blabla, ce que l'on veut dire et écrire. A remplacer. Ici du texte et du blabla, ce que l'on veut dire et écrire. A remplacer. Ici du texte et du blabla, ce que l'on veut dire et écrire. A remplacer. Ici du texte et du blabla, ce que l'on veut dire et écrire. A remplacer. Ici du texte et du blabla, ce que l'on veut dire et écrire. A remplacer.

\subsubsection{Titre de la sous sous section}

Ici du texte et du blabla, ce que l'on veut dire et écrire. A remplacer. Ici du texte et du blabla, ce que l'on veut dire et écrire. A remplacer. Ici du texte et du blabla, ce que l'on veut dire et écrire. A remplacer. Ici du texte et du blabla, ce que l'on veut dire et écrire. A remplacer. Ici du texte et du blabla, ce que l'on veut dire et écrire. A remplacer. Ici du texte et du blabla, ce que l'on veut dire et écrire. A remplacer.

Ici du texte et du blabla, ce que l'on veut dire et écrire. A remplacer. Ici du texte et du blabla, ce que l'on veut dire et écrire. A remplacer. Ici du texte et du blabla, ce que l'on veut dire et écrire. A remplacer. Ici du texte et du blabla, ce que l'on veut dire et écrire. A remplacer. Ici du texte et du blabla, ce que l'on veut dire et écrire. A remplacer. Ici du texte et du blabla, ce que l'on veut dire et écrire. A remplacer.

\subsubsection{Titre de la sous sous section}

Ici du texte et du blabla, ce que l'on veut dire et écrire. A remplacer. Ici du texte et du blabla, ce que l'on veut dire et écrire. A remplacer. Ici du texte et du blabla, ce que l'on veut dire et écrire. A remplacer. Ici du texte et du blabla, ce que l'on veut dire et écrire. A remplacer. Ici du texte et du blabla, ce que l'on veut dire et écrire. A remplacer. Ici du texte et du blabla, ce que l'on veut dire et écrire. A remplacer.

Ici du texte et du blabla, ce que l'on veut dire et écrire. A remplacer. Ici du texte et du blabla, ce que l'on veut dire et écrire. A remplacer. Ici du texte et du blabla, ce que l'on veut dire et écrire. A remplacer. Ici du texte et du blabla, ce que l'on veut dire et écrire. A remplacer. Ici du texte et du blabla, ce que l'on veut dire et écrire. A remplacer. Ici du texte et du blabla, ce que l'on veut dire et écrire. A remplacer.

\subsection{Conclusion}

Ici du texte et du blabla, ce que l'on veut dire et écrire. A remplacer. Ici du texte et du blabla, ce que l'on veut dire et écrire. A remplacer. Ici du texte et du blabla, ce que l'on veut dire et écrire. A remplacer. Ici du texte et du blabla, ce que l'on veut dire et écrire. A remplacer. Ici du texte et du blabla, ce que l'on veut dire et écrire. A remplacer. Ici du texte et du blabla, ce que l'on veut dire et écrire. A remplacer.

Ici du texte et du blabla, ce que l'on veut dire et écrire. A remplacer. Ici du texte et du blabla, ce que l'on veut dire et écrire. A remplacer. Ici du texte et du blabla, ce que l'on veut dire et écrire. A remplacer. Ici du texte et du blabla, ce que l'on veut dire et écrire. A remplacer. Ici du texte et du blabla, ce que l'on veut dire et écrire. A remplacer. Ici du texte et du blabla, ce que l'on veut dire et écrire. A remplacer.

\subsection{Titre de la sous section}

Ici du texte et du blabla, ce que l'on veut dire et écrire. A remplacer. Ici du texte et du blabla, ce que l'on veut dire et écrire. On peut faire une citation \cite{Motclef1}.
A remplacer. Ici du texte et du blabla, ce que l'on veut dire et écrire. A remplacer. Ici du texte et du blabla, ce que l'on veut dire et écrire. A remplacer. Ici du texte et du blabla, ce que l'on veut dire et écrire. A remplacer. Ici du texte et du blabla, ce que l'on veut dire et écrire. A remplacer.

Ici du texte et du blabla, ce que l'on veut dire et écrire. A remplacer. Ici du texte et du blabla, ce que l'on veut dire et écrire. A remplacer.
Ici du texte et du blabla, ce que l'on veut dire et écrire. A remplacer. Ici du texte et du blabla, ce que l'on veut dire et écrire. A remplacer. Ici du texte et du blabla, ce que l'on veut dire et écrire. A remplacer. Ici du texte et du blabla, ce que l'on veut dire et écrire. A remplacer.

\subsection{Titre de la sous section}

Ici du texte et du blabla, ce que l'on veut dire et écrire. A remplacer. Ici du texte et du blabla, ce que l'on veut dire et écrire. On peut faire une citation \cite{Motclef1}.
A remplacer. Ici du texte et du blabla, ce que l'on veut dire et écrire. A remplacer. Ici du texte et du blabla, ce que l'on veut dire et écrire. A remplacer. Ici du texte et du blabla, ce que l'on veut dire et écrire. A remplacer. Ici du texte et du blabla, ce que l'on veut dire et écrire. A remplacer.

Ici du texte et du blabla, ce que l'on veut dire et écrire. A remplacer. Ici du texte et du blabla, ce que l'on veut dire et écrire. A remplacer.
Ici du texte et du blabla, ce que l'on veut dire et écrire. A remplacer. Ici du texte et du blabla, ce que l'on veut dire et écrire. A remplacer. Ici du texte et du blabla, ce que l'on veut dire et écrire. A remplacer. Ici du texte et du blabla, ce que l'on veut dire et écrire. A remplacer.

\clearpage
