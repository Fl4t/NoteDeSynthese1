\chapter{Gestion de version de code source}

\section{Git : le gestionnaire de version}

Les logiciels de gestion de versions ou VCS\, \footnote{Version Control
System} sont utilisés principalement par les développeurs. En effet, ils sont
quasi exclusivement utilisés pour gérer des codes sources, car ils sont
capables de suivre l’évolution d’un fichier texte ligne de code par ligne de
code. Ces logiciels sont fortement conseillés pour gérer un projet
informatique.

Ils retiennent qui a effectué chaque modification de chaque fichier et
pourquoi. Ils sont par conséquent capables de dire qui a écrit chaque ligne de
code de chaque fichier et dans quel but ; si deux personnes travaillent
simultanément sur un même fichier, ils sont capables d’assembler (de fusionner)
leurs modifications et d’éviter que le travail d’une de ces personnes ne soit
écrasé.

Ces logiciels ont donc par conséquent deux utilités principales :
\begin{itemize}
    \item suivre l’évolution d’un code source, pour retenir les modifications
effectuées sur chaque fichier et être ainsi capable de revenir en arrière en
cas de problème ;
    \item travailler à plusieurs, sans risquer de se marcher sur les pieds.
Si deux personnes modifient un même fichier en même temps, leurs modifications
doivent pouvoir être fusionnées sans perte d’information.
\end{itemize}

\subsection{Les rudiments}

\emph{Un commit} correspond à un enregistrement des modifications dans le temps.
Admettons un fichier contenant un paragraphe, si nous ajoutons un deuxième
paragraphe, le fichier sera considérer modifié par Git\, \footnote{Créé par
Linus Torvalds, qui est entre autres l’homme à l’origine de Linux. Il est de
type distribué}, pour enregistrer la modification on effectue un commit.
L'analogie la plus simple est celle des sauvegardes dans les jeux vidéos ou
vous sauvegardez votre progression.

\emph{Une branche} représente une \og copie virtuelle \fg{} du dossier contenant
les fichiers sources. En effet, il est possible à partir d'un dépôt de le
cloner virtuellement et de basculer de l'un à l'autre pour par exemple
développer sur la branche principale les nouvelles fonctionnalitées et sur
l'autre branche les corrections de bug.

\emph{La fusion} est l'opération de rassemblement des deux branches en une,
c'est-à-dire dans le cas précédent de regrouper les corrections de bug avec les
nouvelles fonctionnalitées.

Les fondamentaux ayant été acquis, j'ai trouvé plusieurs solutions à la
problématique principale qui est la suivante :
\begin{itemize}
    \item Peut-on fusionner des branches tout en choisissant de ne pas
fusionner tout les commits ?
\end{itemize}

La commande \emph{git chery-pick \og nomducommit \fg{}} permet de \og cueillir
\fg{} le commit que l'on veut fusionner, sinon on peut aussi rassembler les deux
branches puis faire un \emph{git revert \og nomducommit \fg{}} pour retirer le
commit après la fusion ou enfin faire trois branches distinctes pour ne pas
avoir à faire ses opérations ci-dessus.

Git répondant au besoin de l'entreprise, il a fallut que j'effectue des
recherches pour que la migration Subversion\, \footnote{Probablement le
logiciel de gestion de versions le plus utilisé à l’heure actuelle. Il est de
type centralisé} vers Git ce fasse sans perte.

\clearpage
