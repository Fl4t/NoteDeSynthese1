\chapter{Gestion de version de code source}

\section{Git : le gestionnaire de version}

Les logiciels de gestion de versions sont utilisés principalement par les
développeurs. En effet, ils sont quasi exclusivement utilisés pour gérer des
codes sources, car ils sont capables de suivre l’évolution d’un fichier texte
ligne de code par ligne de code. Ces logiciels sont fortement conseillés pour
gérer un projet informatique.

Ils retiennent qui a effectué chaque modification de chaque fichier et
pourquoi. Ils sont par conséquent capables de dire qui a écrit chaque ligne de
code de chaque fichier et dans quel but ; si deux personnes travaillent
simultanément sur un même fichier, ils sont capables d’assembler (de fusionner)
leurs modifications et d’éviter que le travail d’une de ces personnes ne soit
écrasé.

Ces logiciels ont donc par conséquent deux utilités principales :
\begin{itemize}
    \item suivre l’évolution d’un code source, pour retenir les modifications effectuées sur chaque fichier et être ainsi capable de revenir en arrière en cas de problème ;
    \item travailler à plusieurs, sans risquer de se marcher sur les pieds. Si deux personnes modifient un même fichier en même temps, leurs modifications doivent pouvoir être fusionnées sans perte d’information.
\end{itemize}

\subsection{Les rudiments}

Un commit correspond à un enregistrement des modifications dans le temps.
Admettons un fichier contenant un paragraphe, si nous ajoutons un deuxième
paragraphe, le fichier sera considérer modifié par Git, pour enregistrer la
modification on effectue un commit. L'analogie la plus simple est celle des
sauvegardes dans les jeux vidéos ou vous sauvegardez votre progression.

Une branche représente une copie virtuelle du dossier contenant les fichiers
sources. En effet, il est possible a partir d'un dépôt de cloner virtuellement
ce dépôt et de basculer de l'un à l'autre pour par exemple développer sur la
branche principale les nouvelles fonctionnalités et sur l'autre branche les
corrections de bug.

La fusion est l'opération rassemblement des deux branches en une, c'est-à-dire
dans le cas précédent de regrouper les corrections de bugs avec les nouvelles
fonctionnalités.

Les fondamentaux ayant été acquis, j'ai trouvé plusieurs solutions à la problématique principale qui est la suivante :
\begin{itemize}
    \item Peut-on fusionner des branches tout en choisissant de ne pas fusionner tout les commits ?
\end{itemize}

La commande git chery-pick "commit" permet de \fg cueillir \og le commit que
l'on veut fusionner, sinon on peut aussi rassembler les deux branches puis faire un
git revert "commit" pour retirer le commit après la fusion ou enfin faire trois branches distinctes pour ne pas avoir à faire ses opérations ci-dessus.

Git répondant au besoin de l'entreprise, il a fallut que j'effectue des recherches pour que la migration Subversion vers git ce fasse sans perte.

%-- Note de bas de page sur les stades
%\protect\footnote{Par exemple, on peut faire un pied de page :
%\begin{itemize}
%\item avec une liste à puces ;
%\item avec une liste à puces ;
%\item avec une liste à puces.
%\end{itemize}
}
%-- Fin Note de bas de page sur les stades

\clearpage
