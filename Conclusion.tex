\chapter{Conclusion} % (fold)
\label{cha:Conclusion}

\lettrine{L}{a} première chose qui me vient à l'esprit est la
progression que m'a apporté ce stage. Je peux dire en toute humilité que
j'en ressors grandi.  L'utilitaire de comparaison m'a donné beaucoup de
fil à retordre car à chaque fois qu'il fonctionnait, \bsc{M.~Dubourg} me
poussait à être de plus en plus critique sur ce que je faisais et à
recommencer en voyant les choses de manière différente.  J'ai pu
connaître enfin le travail de développeur en entreprise et je dois dire
que cela me plait beaucoup, cependant ce n'est pas toujours évident.
Bloquer sur un problème toute la journée sans avancer est très pénible
intellectuellement, j'ai connu beaucoup de maux de tête et de matins
difficiles. Travailler toute la journée à concevoir des algorithmes est
éprouvant, et encore je ne subissais aucune pression quant à ma
productivité\dots

J'ai été très satisfait des échanges que j'ai pu avoir avec l'équipe.
Développer est une chose mais proposer, débattre, trouver les meilleures
solutions sont ce que j'ai préféré. L'entreprise m'a beaucoup apporté,
mais je pense aussi avoir apporté des choses et c'est très gratifiant.

Cette période m'a prouvé que je ne m'étais pas trompé dans mon
orientation. Par contre, je suis maintenant indécis pour ma poursuite
d'étude. Si je continue, ma passion pour les logiciels libres me pousse
vers la licence professionnelle \og logiciels libres et propriétaires
pour systèmes, réseaux et bases de données \fg{}, par contre ma force de
proposition m'orienterait plus vers la \og spécialité assistant chef de
projet informatique \fg{}.
% chapter Conclusion (end)
