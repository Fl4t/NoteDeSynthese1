\chapter{Conclusion}

La première chose qui me vient à l'esprit et l'enrichissement que m'a apporté
ce stage. Je peux dire en toute humilité que j'en ressors grandi. L'utilitaire
de comparaison ma donnée beaucoup de fil a retordre car à chaque fois qu'il
fonctionnait, M.\bsc{Dubourg} me poussait à être de plus en plus critique sur
ce que je faisais et a recommencé en voyant les choses de manière différente.
J'ai pu connaitre enfin le travail de développeur en entreprise et je dois dire
que ça me plait beaucoup cependant ce n'est pas toujours évident. Bloquer sur
un problème toute la journée sans avancer est très pénible intellectuellement,
j'ai connu beaucoup de mots de tête et de matin difficile. Travailler du matin
au soir à concevoir des algorithmes est éprouvant et encore, je ne subissais
aucune pression quant à ma productivité \ldots{}

J'ai été très satisfait quand aux échanges que j'ai pu avoir avec l'équipe.
Développer est une chose mais proposer, débattre, trouver les meilleures
solutions sont ce que j'ai préféré. L'entreprise m'a beaucoup apporté, mais je
pense aussi avoir apporté des choses et c'est très gratifiant.

Cette période ma conforté dans mes choix et dans ma poursuite d'étude. Ma
passion pour les logiciels libres me pousse vers La licence professionnelle
Systèmes informatiques et logiciels option logiciels libres et propriétaires
pour les systèmes, réseaux et bases de données, par contre ma force de
proposition m'orienterai plus vers la licence professionnelle Spécialité
Assistant chef de projet informatique.
