\chapter{Introduction}

Je m'appelle Julien \bsc{Stechele}, je suis actuellement en BTS\,
\footnote{\emph{Brevet de Technicien Supérieur.}} informatique de gestion
option développeur d'applications et dans le cadre de mes études j'ai la chance
d'effectuer deux stages durant cette formation. Mon stage de première année
s'est déroulé du 16 mai 2011 au 8 juillet 2011 dans la société \emph{IdentIt}
qui se situe à Petite-Synthe. C'est une SARL\, \footnote{\emph{Société À
Responsabilité Limitée.}} de quatre personnes, M.\bsc{Dubourg} et
M.\bsc{Lesage} sont les fondateurs de cette structure, le premier s'occupe de
la partie technique en tant que chef de projet et le second de la partie
gestion de l'entreprise. Ils sont accompagnés par deux développeurs, l'un
travaillant sur la partie application sur Windows Mobile \copyright, l'autre
sur la partie internet.

Mon premier travail consistait à développer un utilitaire pour la partie web de
comparaison de base de données qui permetterait d'améliorer le suivi des mises
à jour d'une base obsolète à partir d'une base de référence et aussi de fournir
les requêtes permettant cette mise à jour. Mon deuxième travail était de mettre
en place un nouvel outil de gestion de version de code source plus évolué que
l'existant.

Le premier jour en entreprise fut un peu spécial pour moi. En effet, mon
entretien s'étant passé un vendredi après-midi, seuls les gérants de
l'entreprise étaient présents, du coup cela a été pour moi l'occasion de
rencontrer les deux développeurs, Cyril\, \footnote{Titulaire d'un BTS IG à
Jean \bsc{Bart} promotion 2009.} tout juste arrivé dans l'entreprise et
Ludovic, développeur expérimenté qui a de nombreuses années de programmation
derrière lui.

Cette note de synthèse suit un plan spécifique, mais les événements ne ce sont
pas passés dans l'ordre réel. En effet, à cause des difficultés rencontrées et
des problématiques nouvelles données en cours de stage, je suis souvent passé
d'une réalisation à une autre.
