\chapter{Introduction}

Je m'apelle Julien \bsc{Stechele}, je suis actuellement en BTS\,
\footnote{\emph{Brevet de Technicien Supérieur.}} informatique de gestion option
développeur d'applications et dans le cadre de mes études j'ai la chance
d'effectuer deux stages durant cette formation. Mon stage de première année
c'est déroulé du 16 mai 2011 au 8 juillet 2011 dans la société \emph{IdentIt}
qui ce situe à Petite-Synthe. C'est une SARL\, \footnote{\emph{Société à
responsabilité limitée.}} de quatre personnes, M.\bsc{Dubourg} et M.\bsc{Lesage}
sont les fondateurs de cette structure, le premier s'occupant de la partie
technique entant que chef de projet, le second s'attelant plus de la partie
gestion de l'entreprise. Ils sont accompagnés par deux développeurs, l'un
travaillant sur la partie application sur Windows Mobile \copyright, l'autre
sur la partie internet.

Mon travail consistait à développer un script sur la partie web de comparaison
de base de donnée mais aussi de mettre en place un nouvel outil de gestion de
version de code source plus évolué que l'existant. Le but étant d'améliorer le
suivi des mise à jours d'une base de référence vers une base obsolète mais
aussi de fournir les requêtes SQL\, \footnote{\emph{Structured Query Language}
est un langage informatique normalisé qui sert à effectuer des opérations sur
des bases de données} permettant cette mise à jour.

Le premier jour en entreprise fut un peu spécial pour moi. En effet, mon
entretien s'étant passé un vendredi après-midi, seul les gérants de
l'entreprise était présent, du coup ça à été pour moi l'occasion de rencontrer
les deux développeurs, Cyril\, \footnote{Titulaire d'un BTS IG à jean bart
promotion 2009.} tout juste arrivé dans l'entreprise et Ludovic, développeur
expérimenté qui a connu plusieurs entreprises est à de nombreuses années de
programmation derrière lui.\\ Cette note de synthèse suivant un plan
spécifique, les événements ne ce sont pas passé dans l'ordre exacte dans lequel
je l'explique. En effet, à cause des difficultés rencontrés est des
problématiques nouvelles donné en cour de stage, je suis beaucoup passer d'une
réalisation à une autre.  Dans un premier temps M.Dubourg m'a proposer de me
documenter un maximum sur Git pour que je maitrise globalement l'application
pour savoir si elle répondait bien aux attentes, c'est ce que je présenterai en
première partie.

\clearpage
