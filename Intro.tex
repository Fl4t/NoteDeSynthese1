\chapter{Introduction}

Je m'apelle Julien \bsc{Stechele}, je suis actuellement en Brevet de Technicien suppérieur en informatique de gestion option développeur d'applications et dans le cadre de
mes études j'ai la chance d'éffectuer deux stages durant cette formation. Mon stage de première année c'est éffectué du 16 mai 2011 au 8 juillet 2011 dans la société IdentIt
qui ce situe a Petite-Synthe. C'est une SARL de quatres personnes, M.\bsc{Dubourg} et M.\bsc{Lesage} sont les fondateurs de cette structure, le premier s'occupant de la
partie technique entant que chef de projet, le deuxième s'attelant plus de la partie gestion de l'entreprise. Ils sont accompagné par deux développeurs, l'un travaillant sur
la partie application sur Windows Mobile, l'autre sur la partie internet.

Mon travail consistait à développer un script sur la partie web de comparaison de base de donnée mais aussi de mettre en place un nouvel outil de gestion de version de code
source plus évolué que l'éxistant. Le but étant d'améliorer le suivi des mise à jours d'une base de référence vers une base obsolète mais aussi de fournir les requêtes SQL
permettant cette mise à jour.

Le premier jour en entreprise fut un peu spécial pour moi. En effet, mon entretien s'étant passé un vendredi après-midi, seul les gérants de l'entreprise était présent, du
coup ça à été pour moi l'occasion de rencontrer les deux développeurs, Cyril (ancien BTS IG a jean bart) jeune employé et Ludovic, développeur experimenté qui a connu
plusieurs entreprises à de nombreuses années de programmation derrière lui.

Cette note de synthèse suivant un plan spécifique, les événements ne ce sont pas passé dans l'ordre exacte dans lequel je l'explique. En effet, à cause des difficultés
rencontrés est des problèmatiques nouvelles donnée en cour de stage, je suis beaucoup passer d'une réalisation à une autre.

Dans un premier temps M.Dubourg ma proposer de me documenter un maximum sur Git pour que je maitrise globalement l'application et savoir si elle
répondais bien aux attentes, c'est ce que je présenterai en première partie.

\clearpage
