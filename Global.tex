\documentclass[12pt,a4paper,utf8x]{report}
\usepackage [frenchb]{babel}

% Pour pouvoir utiliser
\usepackage{ucs}
\usepackage[utf8x]{inputenc}

\usepackage{url} % Pour avoir de belles url
\usepackage {geometry}

% Pour mettre du code source
\usepackage {listings}
% Pour pouvoir passer en paysage
\usepackage{lscape}

% Pour pouvoir faire plusieurs colonnes
\usepackage {multicol}
% POur crééer un index
\usepackage{makeidx}
\makeindex

% Pour les entetes de page
% \usepackage{fancyheadings}
%\pagestyle{fancy}
%\renewcommand{\sectionmark}[1]{\markboth{#1}{}}
%\renewcommand{\subsectionmark}[1]{\markright{#1}}

% Pour l'interligne de 1.5
\usepackage {setspace}
% Pour les marges de la page
\geometry{a4paper, top=2.5cm, bottom=3.5cm, left=1.5cm, right=1.5cm, marginparwidth=1.2cm}

\parskip=5pt %% distance entre § (paragraphe)
\sloppy %% respecter toujours la marge de droite

% Pour les pénalités :
\interfootnotelinepenalty=150 %note de bas de page
\widowpenalty=150 %% veuves et orphelines
\clubpenalty=150

%Pour la longueur de l'indentation des paragraphes
\setlength{\parindent}{15mm}



%%%% debut macro pour enlever le nom chapitre %%%%
\makeatletter
\def\@makechapterhead#1{%
  \vspace*{50\p@}%
  {\parindent \z@ \raggedright \normalfont
    \interlinepenalty\@M
    \ifnum \c@secnumdepth >\m@ne
        \Huge\bfseries \thechapter\quad
    \fi
    \Huge \bfseries #1\par\nobreak
    \vskip 40\p@
  }}

\def\@makeschapterhead#1{%
  \vspace*{50\p@}%
  {\parindent \z@ \raggedright
    \normalfont
    \interlinepenalty\@M
    \Huge \bfseries  #1\par\nobreak
    \vskip 40\p@
  }}
\makeatother
%%%% fin macro %%%%

%Couverture

\title
{
	\normalsize{Lycée Jean Bart - Dunkerque\\
	2011-2012}\\
	\vspace{15mm}
	\Huge{Note de synthèse}
}
\author{\bsc{Stechele} Julien\\
	\vspace{45mm}
}

\date{
	\normalsize{IdentIt\\
    1294 rue Achille \bsc{Pérès}\\
	Petite-Synthe\\
	\vspace{5mm}
    Tuteur de stage : M.\bsc{Anselin}\\
	Maitre de stage : M.\bsc{Dubourg}
	}
}

\begin{document}

\maketitle

Remerciements

Je tiens à remercier :

M.\bsc{Dubourg} pour le suivi qu'il m'a apporté pendant toute la durée du stage, le temps qu'il ma consacrer pour m'initier à la programmation orienté objet, les astuces techniques pour développer plus rapidement ainsi que l'enseignement des habitudes propres à l'entreprise.

L'entreprise IdentIt pour avoir accepter ma candidature, j'espère avoir été a la hauteur de leurs attentes.

L'équipe de développeur pour m'avoir aiguiller quand j'étais en détresse.

L'équipe enseignante aussi pour nous avoir appris les fondements du développement et l'univers informatique tout autour qui en découle.



\tableofcontents
\clearpage

% Pour avoir un interligne de 1,5
\begin{onehalfspace}

\chapter{Introduction}

Je m'appelle Julien \bsc{Stechele}, je suis actuellement en BTS\,
\footnote{\emph{Brevet de Technicien Supérieur.}} informatique de gestion
option développeur d'applications et dans le cadre de mes études j'ai la chance
d'effectuer deux stages durant cette formation. Mon stage de première année s'est déroulé du 16 mai 2011 au 8 juillet 2011 dans la société \emph{IdentIt}
qui se situe à Petite-Synthe. C'est une SARL\, \footnote{\emph{Société à
responsabilité limitée.}} de quatre personnes, M.\bsc{Dubourg} et
M.\bsc{Lesage} sont les fondateurs de cette structure, le premier s'occupant de
la partie technique entant que chef de projet, le second s'attelant de la
partie gestion de l'entreprise. Ils sont accompagnés par deux développeurs,
l'un travaillant sur la partie application sur Windows Mobile \copyright,
l'autre sur la partie internet.

Mon premier travail consistait à développer sur la partie web un utilitaire de
comparaison de base de données qui permetterait d'améliorer le suivi des mises
à jour d'une base obsolète à partir d'une base de référence et aussi de fournir
les requêtes permettant cette mise à jour. Mon deuxième travail était de mettre
en place un nouvel outil de gestion de version de code source plus évolué que
l'existant.

Le premier jour en entreprise fut un peu spécial pour moi. En effet, mon
entretien s'étant passé un vendredi après-midi, seul les gérants de
l'entreprise était présent, du coup cela a été pour moi l'occasion de
rencontrer les deux développeurs, Cyril\, \footnote{Titulaire d'un BTS IG à
Jean \bsc{Bart} promotion 2009.} tout juste arrivé dans l'entreprise et
Ludovic, développeur expérimenté qui a de nombreuses années de programmation
derrière lui.

Cette note de synthèse suivant un plan spécifique, les événements ne ce sont
pas passés dans l'ordre exacte dans lequel je les raconte. En effet, à cause
des difficultés rencontrées et des problématiques nouvelles données en cours de
stage, je suis beaucoup passé d'une réalisation à une autre.

\clearpage


\chapter{Gestion de version de code source}

\section{Git : le gestionnaire de version}

Les logiciels de gestion de versions ou VCS\, \footnote{\emph{Version Control
System.}} sont utilisés principalement par les développeurs. En effet, ils sont
quasi exclusivement utilisés pour gérer des codes sources, car ils sont
capables de suivre l’évolution d’un fichier texte \emph{ligne de code par ligne de
code.} Ces logiciels sont fortement conseillés pour gérer un projet
informatique.

Ils retiennent qui a effectué chaque modification de chaque fichier et
pourquoi. Ils sont par conséquent capables de dire qui a écrit chaque ligne de
chaque fichier et dans quel but ; si deux personnes travaillent simultanément
sur un même fichier, ils sont capables d’assembler (de fusionner) leurs
modifications et d’éviter que le travail d’une de ces personnes ne soit écrasé.

Ces logiciels ont donc par conséquent deux utilités principales :
\begin{itemize}
    \item suivre l’évolution d’un code source, pour retenir les modifications
effectuées sur chaque fichier et être ainsi capable de revenir en arrière en
cas de problème ;
    \item travailler à plusieurs, sans risquer de se marcher sur les pieds.
Si deux personnes modifient un même fichier en même temps, leurs modifications
doivent pouvoir être fusionnées sans perte d’information.
\end{itemize}

\subsection{Les rudiments}

\emph{Un commit} correspond à un enregistrement des modifications dans le
temps.  Admettons un fichier contenant un paragraphe, si nous ajoutons un
deuxième paragraphe, le fichier sera considérer comme modifié par Git\,
\footnote{Créé par Linus Torvalds, qui est entre autres l’homme à l’origine de
Linux. Il est de type distribué.}, pour enregistrer la modification on effectue
un commit.  L'analogie la plus simple est celle des jeux vidéos ou vous
sauvegardez votre progression à chaque étape franchie.

\emph{Une branche} représente une \og copie virtuelle \fg{} du dossier contenant
les fichiers sources. En effet, il est possible à partir d'un dépôt de le
cloner virtuellement et de basculer de l'un à l'autre pour par exemple
développer sur la branche principale les nouvelles fonctionnalitées et sur
l'autre branche les corrections de bug.

\begin{center}
\includegraphics{images/branches.png}
\end{center}

\emph{La fusion} est l'opération de rassemblement des deux branches en une,
c'est-à-dire dans le cas précédent de regrouper les corrections de bug avec les
nouvelles fonctionnalitées.

Les fondamentaux ayant été acquis, j'ai trouvé plusieurs solutions à la
problématique principale qui est la suivante :\\ \quote{Peut-on fusionner des
branches tout en choisissant de ne pas fusionner tout les commits ?}\\

\begin{itemize}
    \item Tout dabord la commande \emph{git chery-pick \og nomducommit \fg{}}
        permet de \og cueillir \fg{} le commit que l'on veut fusionner.
    \item Ensuite on peut aussi rassembler les deux branches puis faire un
        \emph{git revert \og nomducommit \fg{}} pour retirer le commit après la
        fusion.
    \item Enfin faire trois branches distinctes pour ne pas avoir à faire les
        opérations ci-dessus.
\end{itemize}

Git répondant au besoin de l'entreprise, il a fallut que j'effectue des
recherches pour que la migration Subversion\, \footnote{Le logiciel de gestion
de versions le plus utilisé à l’heure actuelle. Il est de type centralisé.}
vers Git ce fasse sans perte.

\subsection{L'installation} % (fold)
\label{sec:L'installation}

A peu près au milieu du stage, nous nous sommes heurté a un problème technique.
Le serveur de l'entreprise ne contenai pas de version réçente de Git. En
sachant que celui-ci est mutualisé\, \footnote{L'hébergement mutualisé est un
concept d'hébergement internet destiné principalement à des sites web, dans un
environnement technique dont la caractéristique principale est d'être partagé
par plusieurs utilisateurs. L'administration du ou des serveurs est assurée par
un intervenant tiers tel qu'OVH.}, nous n'avions pas les autorisations
nécessaire pour installer une nouvelle version.

Les logiciels libres sont réputé pour être rétro-compatibles mais en toute
logique il ne dispose pas des nouveautés sans les mettre à jour. Le problème ce
situe sur le fait de pouvoir modifier des branches distantes dites "de suivi".
Cette fonctionnalité permet à plusieurs développeurs de travailler en même
temps sur une branche différente de celle par defaut, par exemple pour
experimenter de nouvelles implémentation à plusieurs. A ce moment là j'ai été
très déçu et pensais mon travail inutilisable. Cependant, mes éfforts n'ont pas
été vain puisque M.Dubourg après des recherches à trouver sur internet un
utilisateur ayant reussi à installer git sur un serveur mutualisé. Nous
n'avions peut-être pas les droits d'accès au dossiers des programmes mais rien
ne nous empechai de compiler git a partir des sources pour nous crée en locale
le logiciel\, \footnote{Le problème ne ce serai pas posé si j'avais lu le
chapitre traitant de la compilation dans mon livre Linux, j'ai découvert cette
méthode qu'après le stage...}. Ceci étant fait, en redéfinissant la variable
système qui stocke les chemins des applications, nous avons pu utilisées notre
compilé dernier cri.
% section L'installation (end)

\subsection{Le passage fatidique} % (fold)
\label{sec:Le passage fatidique}

Trois jours avant mon départ, le grand déménagement s'impose. Toute la partie
étude et confection du tutoriel git prend forme car j'ai basculer tout les
projets qui était au préalable sur Subversion vers Git tout en gardant
l'historique des changement provenant de SVN. Après quelques recherche, une
commande de Git ma permis de répondre à ce besoin.
% section Le passage fatidique (end)

\section{Les utilitaires} % (fold)

Nous avons longuement discuté sur le côté technique de git. En effet cette
outil est à la base un logiciel libre provenant de linux et n'as pas
d'interface graphique, ce choix est établi sur le fait qu'un développeur n'as
pas forcement besoin de ça pour travailler (Sans nul doute que pour un
graphiste, une souris est un élément indispensable pour ses créations mais pas
pour produire du code) et aussi que cette outil regorge de fonctionnalité et
donc très difficile à simplifier à travers une interface ne pouvant fonctionner
qu'avec une souris. L'équipe n'étant pas très féru de ligne de commande et le
logiciel TortoiseGit reprenant que les fonctions éssentielle de Git, il a fallu
que je conçois des petits programmes qui fonctionne en un clic pour simplifier
les choses répétitives.

% section Introduction (end)

\subsection{Lister les dépôts du serveur}

Mon travail suivant consistait a lister les dépôt Git dans une page php. En
effet, le fait que le serveur OVH centralise les codes sources de la société,
les developpeurs devait récuperer une copie de ses dossiers pour pouvoir
travailler dessus, le problème étant qu'il devait lancer FilleZilla pour
connaitre les noms des dossiers Git puis recopier la bonne url avec le bon nom
est très fastidieux. J'ai donc à l'aide de mes connaissance en système UNIX
crée un script en langage PHP qui analyse un dossier défini et qui liste les
differents dépôt tout en leur métant les bonnes addresses de téléchargement en
préfixe. Du coup, un simple copier-collé du lien du dépôt voulu dans
tortoiseGit suffit à cloner. Gain de productivité et de simplicité en somme.

\subsection{Mise à jour locale automatique} % (fold)

Les developpeurs possédent des clones des dépôts distants\, \footnote{Ils ce
trouvent tous sur le serveur OVH qui sert de point de rencontre comme nous
l'avons vu.} en locale pour bien évidament pour maintenir le code source. Le
fait qu'il y ai une multitude de dépôts implique qu'il faut mettre à jour les
clones locaux à chaque fois pour récuperer les modifications des autres
développeurs ce qui est très répétitif et fastidieux. J'ai donc crée un script
Batch\, \footnote{Désigne un fichier contenant une suite de commandes qui
seront traitées automatiquement par Windows \copyright.} qui parcours tout les
dossiers contenant les sources des dépôts et les mets a jour. Pour cela j'ai
parcouru la documentation de la console windows et ça n'as pas été évident du
tout. Pour l'anecdote, je devais à partir de mon MacBook\, \footnote{Système de
type UNIX.}, lancer une machine virtuelle windows\, \footnote{À partir du
logiciel VirtualBox.} pour crée mon script batch qui lance une console Linux
pour faire la mise à jour...

% section Mise à jour distante automatique (end)

\subsection{Mise en production automatique} % (fold)

Une fois que les développeurs sont satisfait de leurs modifications, il doivent
mettre à jour le dépots distant pour partager leurs travaux. Une fois ses
modifications validées par le chef de projet, celui-ci doit mettre en ligne sur
le dépôt de production. J'ai crée un script Batch pour que ça ce fasse en un
clic.

% section Mise en production automatique (end)
\clearpage


\chapter{Script de comparaison}

N'ayant pas de notions de programmation orientée objet\, \footnote{Ce qui sera
enseignée en deuxième année.} je n'ai pas pu rejoindre les développeurs de
l'entreprise dans l'application qu'ils développaient, cela dit on m'a confié
une tâche annexe qui est la comparaison des bases de données. Le schéma
\ref{bdd} en page \pageref{bdd} nous en illustre une structure basique.

\begin{figure}
\begin{center}
\includegraphics[scale=0.5]{images/bdd.png}
\end{center}
\caption{Un schéma de base de données simple.}
\label{bdd}
\end{figure}

Le besoin premier de ce script est de consulter les différences qui existent
entre une base de référence est une base à mettre à jour. Dans le cas de
l'entreprise, il permettrait un suivi des mises à jour des applications
fournies au client et pour moi, m'initier à la programmation orientée objet\,
\footnote{La programmation orientée objet est un paradigme de programmation qui
consiste à utiliser des objets ; un objet représente un concept, une idée ou
toute entité du monde physique, comme une voiture, une personne ou encore une
page d'un livre.} dès le premier stage.

\section{Le départ}

À l'aide d'un cours sur internet, j'ai commencé à créer ma première classe.
Cette classe après instanciation représenterait l'objet \og base de données
\fg{} sous forme de tableau. L'image \ref{obj} qui ce trouve en page
\pageref{obj} est plus parlante.

\begin{figure}
\begin{center}
\includegraphics[scale=0.5]{images/objet.png}
\end{center}
\caption{Un plan à partir duquel on crée des objets.}
\label{obj}
\end{figure}

Passer de la programmation fonctionnelle\, \footnote{La programmation
fonctionnelle est un paradigme de programmation qui repose sur l'utilisation
majoritaire des fonctions et procédures.} à l'objet fut vraiment difficile. Me
rendant compte que je bloquais énormément, je fis des recherches sur internet
pour trouver un programme équivalent sur lequel je me suis appuyé pour
commencer. M.\bsc{Dubourg} m'a mis sur la voie en me disant d'utiliser des
classes et des méthodes toutes prêtes de l'entreprise ce qui me fit gagner
beaucoup de temps, car je savais pas comment transcrire une structure de base
de données en objet.

Pour utiliser les sources de la société, j'ai dû mettre en place un répertoire
de travail et récupérer les sources via leur ancien gestionnaire de version de
code source. Ceci étant fait il s'agissait maintenant de réussir à faire
fonctionner le site web principal en local sur ma machine. Comme mon logiciel
MAMP\, \footnote{\emph{Macintosh Apache MySQL PHP.} est une combinaison de
logiciel.} m'affichait plein d'erreurs, j'ai décidé d'installer manuellement
chacun des logiciels présents dans celui-ci. Même après cette manipulation le
problème n'avait pas disparu, mon tuteur de stage m'est venu en aide après de
longues heures à chercher en vain. Le problème était tout d'abord au niveau de
la configuration de PHP, qui affichait tous les avertissements de manière trop
stricte, ce qui entravait le lancement de la page principale. Il était aussi au
niveau du cache de mon navigateur. En effet, j'ai du vider celui-ci pour faire
enfin apparaitre le site proprement. De nombreuses heures de recherche juste à
cause d'un cache internet non vidé fut extrêmement frustrant \ldots{}

\section{Comparaison des tables}

Il a fallu dans un premier temps que je recherche comment extraire le nom d'une
base ainsi que le nom de ses tables. La réponse à cette question est dans la
documentation de MySQL. Une base de données est fournit à l'installation et
s'appelle \og information schéma \fg{} . En résumé, c'est une base de données
qui contient toutes les autres.

Ensuite, j'ai transformé le programme procédural sur plusieurs semaines en
objet, le fait de passer par cette étape intermédiaire m'a permis d'abord de
résoudre le problème algorithmiquement, puis de me consacrer sur la façon de
l'écrire.

La requête récupérant les données utiles, j'ai dû concevoir un algorithme
capable de comparer les deux tableaux d'objets par rapport à leurs noms :
\begin{itemize}
    \item Si la base de données de référence a une table qui n'est pas dans
    la base de données à mettre à jour, on l'ajoute ;
    \item Si la base de données à mettre à jour contient une table qui n'est
    pas dans la base de données de référence, on l'enlève ;
    \item Si les tables comparées sont toutes les deux dans les bases de
    données, on compare l'intérieur des tables.
\end{itemize}

J'ai dû revoir mon algorithme plusieurs fois, car la fonction de comparaison de
chaine de caractère \texttt{strnatcmp} compare les mots comme un humain le
ferait, alors que mySQL trie les tables de manière binaire grâce aux codes
ASCII\, \footnote{\emph{American Standard Code for Information Interchange.} ou
\og Code américain normalisé pour l'échange d'information \fg{} est la norme de
codage de caractères en informatique la plus connue, la plus ancienne et la
plus largement compatible. ASCII contient les caractères nécessaires pour
écrire en anglais.} des caractères ce qui faisait que la première fonction
fonction donnait des résultats erronés comme le montre le tableau \ref{tab}.

\begin{table}
\begin{center}
\begin{tabular}{|c|c|}
\hline
\textbf{Tri de chaînes standard} & \textbf{Tri de chaînes ordre naturel} \\
\hline
[0] = img1.png & [0] = img1.png \\
\hline
[1] = img10.png & [1] = img2.png \\
\hline
[2] = img12.png & [2] = img10.png \\
\hline
[3] = img2.png & [3] = img12.png \\
\hline
\end{tabular}
\caption{La différence entre les tris de chaînes.}
\label{tab}
\end{center}
\end{table}

\section{Comparaison des champs}

J'étais dans la situation des tables ayant les mêmes noms, il fallait que je
compare le contenu à partir du nom des champs. Il ne m'a pas fallut longtemps
pour comprendre que l'algorithme était exactement le même, le plus dur étant de
savoir comment j'allais faire pour ne pas me répéter, ce qui a bloqué
énormément mon avancé pour pas grand-chose. M.\bsc{Dubourg} constatant que je
stagnais, m'a conseillé de faire comme j'avais appris plutôt que d'essayer de
faire de la POO\, \footnote{\emph{Programmation Orientée Objet.}}.  Une fois le
script fonctionnel, qui était cependant mal optimisé et peu lisible, mon tuteur
me guida avec des explications poussées sur la marche à suivre. De fil en
aiguille le code source passa d'environ trois-cent cinquante lignes à cent
cinquante.

Ensuite, je devais comparer le contenu des champs des tables lorsque les champs
parcourus avait le même nom. Je comparais tout simplement le contenu des champs
pour connaitre l'issue finale, à savoir que je ne pouvais pas décider si les
tables étaient similaire sans avoir balayé tous les champs et toutes les
caractéristiques de ceux-ci.

\section{Affichage du résultat}

Après tout ceci fini et optimisé, j'implémente une nouvelle fonctionnalité qui
permettrait d'afficher un texte lisible qui énonce les différences des deux
bases de données. En fonction des résultats obtenus, je génère des balises HTML\,
\footnote{L’ \emph{Hypertext Markup Language} est un langage de balisage conçu
pour représenter les pages web.} et du texte pour que cela soit compréhensible
à l'utilisateur. Ceci étant fait assez rapidement je suis passé à la génération
des requêtes SQL\, \footnote{\emph{Structured Query Language} est un langage
informatique normalisé qui sert à effectuer des opérations sur des bases de
données.} permettant de mettre à jour la base obsolète depuis la base de
référence. C'est à ce stade que j'ai constaté que je ne pourrais pas prendre en
compte les clés étrangères dans mon code à moins de revoir totalement tout le
script. M.\bsc{Dubourg} m'a rassuré sur le fait que l'entreprise n'utilisait pas
le type de base de données myISAM\, \footnote{L'organisation séquentielle
indexée, ou \emph{Indexed Sequential Access Method} en anglais, est un mode
d'organisation des fichiers dans les bases de données.} et que donc aucune de
leurs tables comportait de clé étrangère, cependant les clés primaires
concaténées restent problématiques, car pas imaginer pendant la conception. Nous
avons décidé que la création de ce genre de cas se fera à la main pour ne pas
repartir de zéro.

Après avoir analysé mes méthodes de génération de phrases lisibles et de
requêtes SQL, M.\bsc{Dubourg} m'a présenté un outil nommé \og Smarty \fg{} qui
permet de dissocier la partie traitement de la partie affichage, car il est
vrai que mes méthodes étaient vraiment très sales. Pour résumer, j'écrivais des
chaînes de caractères dans une seule variable en écrivant les noms des tables
ou des champs, en concaténant le tout avec des balises HTML de saut de ligne un
peu n'importe où. J'ai dû parcourir la documentation de Smarty pour comprendre
son fonctionnement, il s'avère que la syntaxe est très particulière mais très
efficace. J'ai terminé par faire de l'agencement sur ma page pour que chaque
ligne lisible par un néophyte soit en face de la ligne traduite en SQL dans un
tableau à deux colonnes.

\clearpage


\chapter{Les utilitaires}

\section{Introduction} % (fold)

Nous avons longuement discuté sur le côté technique de git. En effet cette
outil est à la base un logiciel libre provenant de linux et n'as pas
d'interface graphique, ce choix est établi sur le fait qu'un développeur n'as
pas forcement besoin de ça pour travailler (Sans nul doute que pour un
graphiste, une souris est un élément indispensable pour ses créations mais pas
pour produire du code) et aussi que cette outil regorge de fonctionnalité et
donc très difficile à simplifier à travers une interface ne pouvant fonctionner
qu'avec une souris. L'équipe n'étant pas très féru de ligne de commande et le
logiciel TortoiseGit reprenant que les fonctions éssentielle de Git, il a fallu
que je conçois des petits programmes qui fonctionne en un clic pour simplifier
les choses répétitives.

% section Introduction (end)

\section{Lister les dépôts du serveur}

Mon travail suivant consistait a lister les dépôt Git dans une page php. En
effet, le fait que le serveur OVH centralise les codes sources de la société,
les developpeurs devait récuperer une copie de ses dossiers pour pouvoir
travailler dessus, le problème étant qu'il devait lancer FilleZilla pour
connaitre les noms des dossiers Git puis recopier la bonne url avec le bon nom
est très fastidieux. J'ai donc à l'aide de mes connaissance en système UNIX
crée un script en langage PHP qui analyse un dossier défini et qui liste les
differents dépôt tout en leur métant les bonnes addresses de téléchargement en
préfixe. Du coup, un simple copier-collé du lien du dépôt voulu dans
tortoiseGit suffit à cloner. Gain de productivité et de simplicité en somme.

\section{Mise à jour locale automatique} % (fold)

Les developpeurs possédent des clones des dépôts distants\, \footnote{Ils ce
trouvent tous sur le serveur OVH qui sert de point de rencontre comme nous
l'avons vu.} en locale pour bien évidament pour maintenir le code source. Le
fait qu'il y ai une multitude de dépôts implique qu'il faut mettre à jour les
clones locaux à chaque fois pour récuperer les modifications des autres
développeurs ce qui est très répétitif et fastidieux. J'ai donc crée un script
Batch\, \footnote{Désigne un fichier contenant une suite de commandes qui
seront traitées automatiquement par Windows \copyright.} qui parcours tout les
dossiers contenant les sources des dépôts et les mets a jour. Pour cela j'ai
parcouru la documentation de la console windows et ça n'as pas été évident du
tout. Pour l'anecdote, je devais à partir de mon MacBook\, \footnote{Système de
type UNIX.}, lancer une machine virtuelle windows\, \footnote{À partir du
logiciel VirtualBox.} pour crée mon script batch qui lance une console Linux
pour faire la mise à jour...

% section Mise à jour distante automatique (end)

\section{Mise en production automatique} % (fold)

Une fois que les développeurs sont satisfait de leurs modifications, il doivent
mettre à jour le dépots distant pour partager leurs travaux. Une fois ses
modifications validées par le chef de projet, celui-ci doit mettre en ligne sur
le dépôt de production. J'ai crée un script Batch pour que ça ce fasse en un
clic.

% section Mise en production automatique (end)
\clearpage


\input{Partie4}

\chapter{Conclusion}

La première chose qui me viens à l'ésprit et l'enrichissement que ma apporter
ce stage. Je peux dire en toute humilité que j'en ressors grandi. L'utilitaire
de comparaison de base de donnée ma donnée beaucoup de fil a retordre car à
chaque fois qu'il fonctionnais, M.Dubourg me poussai à etre de plus en plus
critique sur ce que je fesai et a recommencer en voyant les choses de manière
différente. J'ai pu connaitre enfin le travail de developpeur en entreprise et
je dois dire que ça me plai beaucoup cependant ce n'est pas toujours évident.
Bloquer sur un problème toute la journée sans avancer est très pénible
intellectuellement, j'ai connu beaucoup de mots de tête et de matin difficile.
Travailler du matin au soirs à concevoir des algorithmes est éprouvant et
encore, je ne subissai aucune pression quand à ma productivité...

J'ai été très satisfait quand aux échanges que j'ai pu avoir avec l'équipe.
Developper est une chose mais proposer, débattre, trouver les meilleurs
solutions sont ce que j'ai préféré. L'entreprise m'a beaucoup apporté mais je
pense aussi avoir apporter des choses et c'est très gratifiant.

Cette periode ma conforté dans mes choix et dans ma poursuite d'étude. Ma
passion pour les logiciels libres me pousse vers La licence professionnelle
Systèmes informatiques et logiciels option logiciels libres et propriétaires
pour les systèmes, réseaux et bases de données, par contre ma force de
proposition m'orienterai plus vers la licence professionnelle Spécialité
Assistant chef de projet informatique.


% Pour finir l'interligne de 1,5
\end{onehalfspace}

\end{document}
